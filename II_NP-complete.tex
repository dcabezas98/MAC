\documentclass{article}
\usepackage[left=2.5cm,right=2.5cm,top=2cm,bottom=2cm]{geometry} % page
                                                             % settings
\usepackage{amsmath} % provides many mathematical environments & tools
\usepackage[spanish]{babel}
\usepackage[doument]{ragged2e}

% Images
\usepackage{graphicx}
\usepackage{float}

\usepackage{enumerate}
\renewcommand{\labelenumii}{\theenumii}
\renewcommand{\theenumii}{\theenumi.\arabic{enumii}.}
\renewcommand{\labelenumiii}{\theenumiii}
\renewcommand{\theenumiii}{\theenumii\arabic{enumiii}.}

% Code
\usepackage{listings}
\usepackage[dvipsnames]{xcolor}
\definecolor{webred}{rgb}{0.5,0.0,0.0}
\newcommand{\n}[1]{{\color{gray}#1}}
\lstset{numbers=left,numberstyle=\small\color{gray}}

\selectlanguage{spanish}
\usepackage[utf8]{inputenc}
\setlength{\parindent}{0mm}

\author{David Cabezas Berrido}
\date{\vspace{-5mm}}

\title{Problema de la Identificación por Intersección\vspace{-2mm}}

\begin{document}
\maketitle

El problema es el siguiente:\\

Dado un conjunto $U$ de tamaño $n$, $m$ subconjuntos
$A_1,\ldots,A_m\subset U$ y $m$ números naturales
$c_1,\ldots,c_m$, determinar si existe un subconjunto $X\subset U$ tal
que en cada intersección $X\cap A_i$ hay $c_i$ elementos: \\
$|X\cap A_i|=c_i\quad\forall i=1,\ldots,m$ \\

Denotaremos este problema como II, veremos que este problema
es NP-completo. \\

Primero comprobaremos que $\text{II} \in \text{NP}$, diseñando un
algoritmo no determinista que lo resuelve. Ante una instancia del
problema II dada por $U$, $A_1,\ldots,A_m$ y
$c_1,\ldots,c_m$:

\begin{enumerate}
\item Para cada uno de los $n$ elementos de $U$, elegimos de forma no
  determinista si ese elemento formará parte de $X$ o no.\qquad\textcolor{webred}{$O(n)$}
\item Para cada uno de los subconjuntos $A_i\subset U\ \ i=1,\ldots,m$):\qquad\textcolor{webred}{$O(m)$}
  \begin{enumerate}
  \item Fijamos $d=0$. \qquad\textcolor{webred}{$O(\log(n))$}
  \item Para cada elemento $a$ de $A_i$: \qquad\textcolor{webred}{$O(n)$}
    \begin{enumerate}
    \item Se comprueba si $a\in X$. \quad\textcolor{webred}{$O(n)$}
    \item En caso afirmativo se incrementa $d$: $d=d+1$. \qquad\textcolor{webred}{$O(\log(n))$}
    \end{enumerate}
  \item Si $d\neq c_i$, se \textbf{rechaza} la entrada. \qquad\textcolor{webred}{$O(\log(n))$}
  \end{enumerate}
\item Se \textbf{acepta} la entrada. \qquad\textcolor{webred}{$O(1)$}
\end{enumerate}

Claramente es un algoritmo que termina siempre, veamos que lo hace en
tiempo polinómico: \\

La línea 1 realiza $n$ pasos, menos que la longitud de la entrada, que
notaremos como $l$. La línea 2 se ejecuta $m\leq l$ veces, en cada una
se recorren todos los elementos de un $A_i$ (a lo más $n$ elementos),
se comprueba si están en $X$ (barrido por a lo más $n$ elementos) y se
incrementa un entero que como mucho vale $n$, su longitud será como
mucho $\log(n)$, luego para incrementarlo se realizarán $O(\log(n))$
pasos, al igual que para fijarlo a 0. La comparación de $d$ con cada
$c_i$ también requerirá $O(\log(n))$ pasos. Luego el número de pasos
para reconocer si una instancia del problema tiene solución es
$O(n)+O(m)\Big(O\big(\log(n)\big)+O(n)\big(O(n)+O\big(\log(n)\big)\big)+O\big(\log(n)\big)\Big)+O(1)$, que es $O(l^3)$. Por tanto $\text{II} \in \text{NP}$.\\

Para ver que es NP-completo, reducimos 3-SAT (que sabemos que es
NP-completo) a II, esta reducción tendrá que ser logarítmica en espacio. \\

Dada una instancia de 3-SAT: un conjunto $P=\{p_1,\ldots,p_k\}$ de
símbolos proposicionales y una colección
$C=\{\gamma_1,\ldots,\gamma_q\}$ de cláusulas sobre los símboles, cada
una con exactamente 3 literales.  Construiremos una instancia del
problema II: $U$, $A_1,\ldots,A_m$ y $c_1,\ldots,c_m$ tal que existe
un subconjunto $X\subset U$ cumpliendo
$|X\cap A_i|=c_i\quad\forall i=1,\ldots,m$ si y solo si existe una
asignación de valores de verdad a los símbolos de $P$ que haga
verdaderas todas las cláusulas de $C$. \\

Tendremos en $U$ todos los símbolos proposicionales y sus negaciones,
en principio $U=\{p_1,\ldots,p_k,\neg p_1,\ldots,\neg p_k\}$.
Entenderemos que seleccionar un símbolo $p_i$ en $X$ es equivalente a
asignarle a ese símbolo el valor de True (1):
$p_i\in X\Leftrightarrow p_i=1$ y
$\neg p_i \in X\Leftrightarrow \neg p_i=1 \Leftrightarrow p_i=0$. \\

Para que realmente la pertenencia al conjunto $X$ represente a un
valor de verdad, debemos forzar que para cada $i=1,\ldots,k$ uno y
sólo uno de los literales $p_i$ y $\neg p_i$ esté en $X$. Esto se
consigue introduciendo conjuntos $A_1,\ldots,A_k$ y naturales
$c_1,\ldots,c_k$ en II tales que
\[A_i=\{p_i,\neg p_i\},\quad c_i=1\qquad \forall i=1,\ldots k\]

Notamos que para cada cláusula de $C$:
$\gamma_i=x\vee y\vee z\quad i=1,\ldots,q$, donde $x,y,z$ son
literales de $P$ (símbolos de $P$ o sus negaciones), la cláusula es
cierta si y solo si alguno de sus literales está en $X$ (es
cierto), esto equivale a que $|X\cap\{x,y,z\}|\geq 1$. Pero en II
tenemos que imponer que el cardinal de la intersección sea un número
exacto, para ello añadimos por cada cláusula $\gamma_i\in C$, un
subconjunto $B_i\subset U$ ($i=1,\ldots,q$) con los literales de la
cláusula como elementos y dos nuevos elementos a $U$: $u_i,v_i$, que
sólo aparecen en el subconjunto asociado a la cláusula $B_i$. \\

De esta forma, si $\gamma_i=x\vee y\vee z$, entonces
$B_i=\{x,y,z,u_i,v_i\}$ e imponemos $|X\cap B_i|=3$. Una asignación de
valores de verdad a los literales $x,y,z$ satisface la cláusula si y
solo si al menos un literal (de $x,y,z$) está en $X$ (es verdadero),
esto equivale a $1\leq|X\cap\{x,y,z\}|\leq 3$ y a que exista una forma
de elegir si $u_i,v_i$ pertenecen o no a $X$ de tal forma que
$|X\cap B_i|=3$: si $|X\cap\{x,y,z\}|=1$, entonces $u_i,v_i\in X$; si
$|X\cap\{x,y,z\}|=2$, entonces $u_i\in X$ y $v_i\notin X$ (o al
revés); y
si $|X\cap\{x,y,z\}|=3$, entonces $u_i,v_i\notin X$. \\

Tomamos
$U=\{p_1,\ldots,p_k,\neg p_1,\ldots,\neg p_k,u_1,\ldots,u_q,v_1,\ldots
v_q\}$, luego $n=2(k+q)$. Tomamos $m=k+q$, reindexamos los
$B_i\ \ i=1,\ldots,q$ como $A_{k+i}\ \ i=1,\ldots,q$ y tomamos
$c_i=3\ \ i=k+1,\ldots,k+q$.\\

Tenemos ahora una instancia del problema II que cumple: existe
$X\subset U$ tal que $|X\cap A_i|=c_i\quad\forall i = 1,\ldots m$ si y
solo si existe una asignación de valores de verdad a los símbolos
proposicionales de la instancia del problema 3-SAT que satisfagan
todas las cláusulas. \\

De existir tal $X$, tenemos la asignación explícitamente consultando
para cada $i=1,\ldots,k$ si es $p_i$ o $\neg p_i$ el que pertenece a
$X$. \\

Recíprocamente, de existir tal asignación es claro que el conjunto de
literales verdaderos interseca en un único elemento con cada
$A_i\quad i=1,\ldots,k$ y se pueden añadir ciertos $u_i$ y $v_i$ a $X$
para que la intersección con cada $A_{i}\quad i=1,\ldots,q$ tenga
exáctamente 3 elementos. \\

Es fácil ver que esta reducción se puede hacer en espacio logarítmico: \\

Primero leemos $P$ y escribimos la primera parte de $U$, con cada
símbolo y su negación. Después leemos $C$ y escribimos la segunda
parte de $U$ con una pareja $u,v$ por cada cláusula. \\

A continuación, leemos $P$ y escribimos los subconjuntos
$A_i\quad i=1,\ldots,k$ cada uno con un símbolo y su negación. Pasamos
otra vez a leer las cláusulas y por cada una escribimos un subconjunto
$A_i\quad i=k+1,\ldots,q$ con los literales de la cláusula y los
símbolos $u_i$ y $v_i$ correspondientes. \\

Finalmente leemos $P$ y por cada elemento escribimos
$c_i=1\quad i=1,\ldots,k$. Y leemos $C$ y por cada cláusula escribimos
$c_i=3\quad i=k+1,\ldots,k+q=m$ en la salida. \\

En ningún momento escribimos en la entrada ni necesitamos volver a
atrás en la salida. Según el modelo, puede que necesitemos almacenar
algunos índices que indiquen por qué símbolo o cláusula vamos leyendo
o escribiendo, pero estos ocupan espacio $O(\log(l))$ donde $l$ es la
longitud de la entrada. \\

Hemos reducido 3-SAT a II usando espacio logarítmico. Por tanto II es
un problema NP-completo.

\end{document}
